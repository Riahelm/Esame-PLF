L’equazione di convezione lineare unidimensionale è un’equazione differenziale alle derivate parziali: 

\begin{equation}
\frac{\partial u}{\partial t} + c \frac{\partial u}{\partial x} = 0
\end{equation}

Per la soluzione numerica di u(x,t) si sono utilizzati i pedici per denotare la posizione spaziale, come u$_i$, e gli apici per denotare l’istante temporale, come un : 

$$
\begin{matrix}
& &\bullet & & \bullet & &  \bullet \\
& &u^{n+1}_{i-1} & & u^{n+1}_i & & u^{n+1}_{i+1} \\
& &\bullet & & \bullet & &  \bullet \\
& &u^n_{i-1} & & u^n_i & & u^n_{i+1} \\
& &\bullet & & \bullet & &  \bullet \\
& &u^{n-1}_{i-1} & & u^{n-1}_i & & u^{n-1}_{i+1} \\
\end{matrix}
$$

L’equazione per fornire la soluzione numerica del problema è data da:

\begin{equation}
u_i^{n+1} = u_i^n - c \frac{\Delta t}{\Delta x}(u_i^n-u_{i-1}^n)
\end{equation}

Le condizioni iniziali per una funzione d’onda quadra sono definite così:

\begin{equation}
u(x,0)=\begin{cases}2 & \text{where } 0.5\leq x \leq 1,\\
1 & \text{everywhere else in } (0, 2)
\end{cases}
\end{equation}

dove il dominio della soluzione numerica è definito in \textit{x} $\in$(0,2).
Poniamo inoltre delle condizioni al contorno su \textit{x}: sia u = 1 quando \textit{x} = 0