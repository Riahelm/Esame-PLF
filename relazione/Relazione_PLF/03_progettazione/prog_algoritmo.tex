\section{Progettazione dell'Algoritmo}
\subsection{Scelte di Progetto}

\subsubsection*{3.1.1 Dati del problema}
Gli insiemi finiti di numeri reali (codomini di ciascuna funzione) si prestano in modo naturale ad essere implementati mediante strutture dati lineari. Queste verranno allocate dinamicamente in quanto non si conosce a priori il numero di punti del dominio di ciascuna funzione, in maniera da non imporre limitazioni rispetto alla specifica del problema.

\subsubsection*{3.1.2 Stampa a schermo}
Per guidare al meglio l'utente durante la fase di acquisizione dei dati, si è scelto di stampare a schermo una descrizione dei parametri utilizzati da ciascun modello di simulazione. Questi sono stati divisi in:
\begin{itemize}
\item parametri iniziali: sono dei valori costanti che vengono utilizzati per impostare le condizioni iniziali di ciascuna simulazione numerica;
\item parametri di simulazione: sono dei valori costanti fissati a priori che limitano l'esperienza di personalizzazione della simulazione numerica dell'utente ai soli dati richiesti in ingresso;
\item parametri richiesti all'utente: sono i dati che l'utente può inserire per ciascuna simulazione. Per ciascun dato viene fornita un'indicazione sui valori più appropriati da digitare. 
\end{itemize} 


\subsection{Passi dell'Algoritmo}
I passi dell'algoritmo per risolvere il problema sono i seguenti:

\begin{enumerate}
\item Acquisire la lunghezza del passo temporale per l'equazione di moto fugoide senza attrito: \textbf{v1.}
\item Calcolare e stampare l'integrazione numerica dell'equazione di moto fugoide senza attrito: \textbf{2.1}.
\item Acquisire la lunghezza del passo temporale per l'equazione di moto fugoide con attrito: \textbf{v1.}
\item Calcolare e stampare l'integrazione numerica dell'equazione di moto fugoide con attrito: \textbf{4.1}.
\item Acquisire il numero di punti totali della funzione d'onda per l'equazione di convezione lineare unidimensionale: \textbf{v2.} e \textbf{5.1}.
\item Calcolare e stampare l'integrazione numerica dell'equazione di convezione lineare unidimensionale: \textbf{6.1}.
\item Acquisire il numero di punti totali della funzione d'onda per l'equazione di Burgers unidimensionale: \textbf{v2.}
\item Calcolare e stampare l'integrazione numerica dell'equazione di Burgers unidimensionale: \textbf{8.1}.
\end{enumerate}

\subsubsection*{2.1 Calcolo del moto fugoide senza attrito} 
\begin{itemize}
\item Alla condizione iniziale $z_0$ si aggiunge il resto del calcolo per l'integrazione numerica dell'equazione:
\begin{itemize}
\item Caso base: se il numero di passi temporali è pari a zero, si effettua un passo di integrazione numerica (vedi metodo di Eulero, in \ref{analisi}, 6) e si restituisce l'altitudine del velivolo $z$.
\item Caso generale: se il numero di passi temporali è maggiore di zero, si effettua un passo di integrazione numerica, si aggiunge l'altitudine del velivolo $z$ alla funzione di traiettoria e poi, una volta decrementato di uno il numero di passi temporali, si procede ricorsivamente sul numero di passi rimanenti. 
\end{itemize}
\end{itemize}

\subsubsection*{4.1 Calcolo del moto fugoide con attrito}
\begin{itemize}
\item Alla condizione iniziale $(x_0,y_0)$ si aggiunge il resto del calcolo per l'integrazione numerica dell'equazione:
\begin{itemize}
\item Caso base: se il numero di passi temporali è pari a zero, si effettua un passo di integrazione numerica (vedi metodo di Eulero, in \ref{analisi}, 17) e si restituisce la coppia di coordinate spaziali $(x,y)$.
\item Caso generale: se il numero di passi temporali è maggiore di zero, si effettua un passo di integrazione numerica, si aggiunge la coppia di coordinate spaziali $(x,y)$ alla funzione di traiettoria e poi, una volta decrementato di uno il numero di passi temporali, si procede ricorsivamente sul numero di passi rimanenti. 
\end{itemize}
\end{itemize}

\subsubsection*{5.1 Acquisizione dati di convezione}
\begin{itemize}
\item Se il numero di punti è pari a zero o uno, viene calcolata solamente la condizione iniziale: \textbf{6.1}
\item Se il numero di punti è maggiore di uno, si acquisisce la lunghezza del passo temporale (\textbf{v1.}), si calcola la condizione iniziale (\textbf{6.1}) e si procede con l'integrazione numerica: \textbf{6.2} 
\end{itemize}

\subsubsection*{6.1 Calcolo della condizione iniziale per l'equazione di convezione}
\begin{itemize}
\item Si genera una lista di punti equidistanti fra loro rappresentante il dominio della funzione d'onda quadra: \textbf{a.} 
\item Si calcola su ogni punto del dominio la funzione d'onda quadra (vedi onda $u$, in \ref{analisi}, 20). 
\end{itemize}

\subsubsection*{6.2 Calcolo dell'integrazione numerica per l'equazione di convezione}
\begin{itemize}
\item Si calcola numericamente l'integrazione della funzione rispetto al tempo:
\begin{itemize}
\item[-] Caso base: se il numero di passi temporali è uguale a zero, viene restituita la funzione d'onda quadra.
\item[-] Caso generale: se il numero di passi temporali è maggiore di zero, si decrementa di uno il numero di passi temporali, si calcola la condizione al contorno (vedi \ref{analisi}, 21)  e la si aggiunge in testa al calcolo numerico dell'integrazione della funzione rispetto allo spazio. Quest'ultima viene effettuata come segue:
\begin{itemize}
\item Caso base: se si raggiunge il numero di passi spaziali totale, si effettua un passo di integrazione numerica
(vedi metodo di Eulero, in \ref{analisi}, 19).
\item Caso generale: se il numero di passi spaziali complessivo non è stato ancora raggiunto, si effettua un passo di integrazione numerica e poi, una volta incrementato di uno il numero di passi spaziali effettuati, si procede ricorsivamente sul numero di passi rimanenti. 
\end{itemize}
\end{itemize}
\end{itemize}

\subsubsection*{8.1 Calcolo dell'equazione di Burgers}
\begin{itemize}
\item Se il numero di punti è zero o uno, l'integrazione numerica dell'equazione è uguale al calcolo della condizione iniziale: \textbf{8.2}.
\item Se il numero di punti è maggiore di uno, si calcola la condizione iniziale dell'equazione (\textbf{8.2}) e si procede con la sua integrazione numerica: \textbf{8.3}. 
\end{itemize}

\subsubsection*{8.2 Calcolo della condizione iniziale per l'equazione di Burgers}
\begin{itemize}
\item Si genera una lista di punti equidistanti fra loro rappresentante il dominio della funzione d'onda a dente di sega: \textbf{a.} 
\item Si applica la condizione iniziale (vedi \ref{analisi}, 25) alla funzione d'onda a dente di sega (vedi onda $u$, in \ref{analisi}, 24). 
\end{itemize}

\subsubsection*{8.3 Calcolo dell'integrazione numerica per l'equazione di Burgers}
\begin{itemize}
\item Si calcola numericamente l'integrazione della funzione rispetto al tempo:
\begin{itemize}
\item[-] Caso base: se il numero di passi temporali è uguale a zero, viene restituita la funzione d'onda a dente di sega.
\item[-] Caso generale: se il numero di passi temporali è maggiore di zero, si decrementa di uno il numero di passi temporali, si calcola la condizione al contorno (vedi \ref{analisi}, 26) e la si aggiunge in testa al calcolo numerico dell'integrazione della funzione rispetto allo spazio. Quest'ultima viene effettuata come segue:
\begin{itemize}
\item Caso base: se si raggiunge il numero di passi spaziali totale, si calcola la condizione di bordo (vedi \ref{analisi}, 26).
\item Caso generale: se il numero di passi spaziali complessivo non è stato ancora raggiunto, si effettua un passo di integrazione numerica (vedi metodo di Eulero, in \ref{analisi}, 23) e poi, una volta incrementato di uno il numero di passi spaziali effettuati, si procede ricorsivamente sul numero di passi rimanenti. 
\end{itemize}
\end{itemize}
\end{itemize}


\subsubsection*{a. Generazione di punti equidistanti}
\begin{itemize}
\item Se il numero di punti da generare è zero, si restituisce una sequenza di punti è vuota.
\item Se il numero di punti da generare è maggiore di zero, si calcola la distanza tra l'estremo superiore e quello inferiore del dominio, si decrementa di uno il numero di punti totali e partendo dall'estremo inferiore del dominio:
\begin{itemize}
\item Caso base: se si è raggiunto il numero massimo di punti da generare, si aggiunge alla sequenza di punti l'estremo inferiore del dominio.
\item Caso generale: se il numero massimo di punti da generare non è stato ancora raggiunto, si aggiunge alla sequenza il punto dell'estremo inferiore del dominio, si incrementa di uno il numero di punti calcolati, si calcola il punto successivo che diventa il nuovo estremo inferiore del dominio e si procede ricorsivamente sul numero di punti da generare rimasti.
\end{itemize}
\end{itemize}

\subsubsection*{v1. Validazione dell'acquisizione della lunghezza del passo temporale}
\begin{itemize}
\item Caso base: se il valore del passo temporale è maggiore di zero, allora il dato viene acquisito.
\item Caso generale: altrimenti si stampa su schermo un messaggio di errore e viene ripetuta l'acquisizione.
\end{itemize}

\subsubsection*{v2. Validazione dell'acquisizione del numero di punti di una funzione}
\begin{itemize}
\item Caso base: se il numero di punti è un intero positivo, il valore viene acquisito.
\item Caso generale: altrimenti si stampa su schermo un messaggio di errore e viene ripetuta l'acquisizione.
\end{itemize}





