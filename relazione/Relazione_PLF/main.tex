
\documentclass{article}

\usepackage[utf8]{inputenc}
\usepackage[T1]{fontenc}
\usepackage{lmodern}
\usepackage[a4paper, textheight=680pt, textwidth=470pt]{geometry}
\usepackage{graphicx}
\usepackage[rightcaption]{sidecap}
\usepackage{caption}
\usepackage{minted}
\usepackage[italian]{babel}
\usepackage{amssymb} %per \nless 
\usepackage{amsmath}

%  /nless è il simbolo "non minore", (< con la striscia)


\begin{document}

\renewcommand\contentsname{Indice}

\begin{titlepage}
	\centering
    \centerline{\includegraphics{Resources/photo5798637185031846732.jpg}}
	\vspace*{\baselineskip}
    \vspace{15\baselineskip}
	\LARGE{\bfseries Relazione relativa al progetto d'esame di Programmazione Logica e Funzionale }\\
	\Large{\ sessione estiva --- a.a. 2023/2024}\\ [0.6cm]
	\large{\textbf{\scshape{Corso di Laurea in Informatica Applicata\\ Università di Urbino}}}\\[3cm]
		{\large {\scshape Studenti:}\\[0.3cm] Barzotti Nicolas\\matricola: 313687\\
        Ramagnano Gabriele\\matricola: 315439}\\
        \vskip 1 cm
	     \large{{\scshape docente:} \\[0.3cm] Marco Bernardo}\\[1.3cm]
%\scshape fa le lettere maiuscole piccole, apprescindere di come sono scritte tra le graffe		
	
\end{titlepage}

\newpage
\tableofcontents
\newpage

\section{Specifica del Problema}

    Scrivere un programma Haskell e un programma Prolog che acquisiscono da tastiera un insieme finito di parametri numerici e poi stampano su schermo il risultato del calcolo numerico per l’integrazione di equazioni differenziali di moto fugoide senza frizione, di moto fugoide con frizione, di convezione lineare a una dimensione ed di Burgers a una dimensione.
    
\section{Analisi del Problema}

\subsection{Dati di Ingresso del Problema}

I dati in ingresso al problema sono stati suddivisi in base alla equazione da integrare numericamente, ne segue quindi la loro descrizione: 

\subsubsection*{Fugoide Senza Attrito}
    Il dato in ingresso è un numero reale maggiore di zero.
    Questo rappresenta il passo temporale per l'integrazione dell'equazione del moto fugoide senza attrito.
\subsubsection*{Fugoide Con Attrito}
    Il dato in ingresso è un numero reale maggiore di zero. Questo rappresenta il passo temporale per l'integrazione dell'equazione del moto fugoide con attrito.
\subsubsection*{Convezione}
    I dati in ingresso per l'integrazione dell'equazione di convezione sono:
    \begin{enumerate}
        \item un numero intero maggiore di zero, rappresenta il numero di punti della funzione d'onda;
        \item un numero reale maggiore di zero, rappresenta la lunghezza del passo temporale.
    \end{enumerate}
\subsubsection*{Burgers}
    I dati in ingresso per l'integrazione dell'equazione di Burgers sono:
    \begin{enumerate}
        \item un numero intero maggiore di zero, rappresenta il numero di punti della funzione d'onda;
        \item un numero reale maggiore di zero, rappresenta la lunghezza del passo temporale.
    \end{enumerate}
\subsection{Dati di Uscita del Problema}

    \subsubsection*{Fugoide Senza Attrito}
    Il dato in uscita all'integrazione dell'equazione del moto fugoide senza attrito è una sequenza di numeri reali che rappresentano la funzione di traiettoria dell'areomobile.
    \subsubsection*{Fugoide Con Attrito}
    Il dato in uscita all'integrazione dell'equazione del moto fugoide con attrito è una sequenza di numeri reali che rappresentano la funzione di traiettoria dell'areomobile.
    \subsubsection*{Convezione}
    Il dato in uscita all'integrazione dell'equazione di convezione a una dimensione è una sequenza di numeri reali che rappresentano i valori finali della funzione d'onda quadrata.
    \subsubsection*{Burgers}
    Il dato in uscita all'integrazione dell'equazione di Burgers a una dimensione è una sequenza di numeri reali che rappresentano i valori finali della la funzione a dente di sega.
\subsection{Relazioni Intercorrenti tra i Dati del Problema}

    \subsubsection*{Fugoide Senza Attrito}
    
L’equazione per il moto di fugoide senza frizione è un’equazione differenziale ordinaria del secondo ordine: 

\begin{equation}
z(t)'' = g - \frac{g \,z(t)}{z_t} = g \left(1 - \frac{z(t)}{z_t}\right)
\end{equation}

Possiamo trasformare questa equazione del secondo ordine in un sistema di equazioni del primo ordine: 

\begin{equation}
z'(t) = b(t)\\
b'(t) = g\left(1-\frac{z(t)}{z_t}\right)
\end{equation}

Un altro modo di considerare un sistema di due equazioni ordinarie del primo ordine è scrivere il sistema differenziale come un’unica equazione vettoriale: 

\begin{equation}
\vec{u}  = \begin{pmatrix} z \\ b \end{pmatrix}
\end{equation}

\begin{equation}
\vec{u}'(t)  = \begin{pmatrix} b\\ g-g\frac{z(t)}{z_t} \end{pmatrix}
\end{equation}

La soluzione approssimativa al tempo tn è un e la soluzione numerica dell’equazione differenziale consiste nel calcolare una sequenza di soluzioni con la seguente formula, basata sull’equazione: 

\begin{equation}
u_{n+1} = u_n + \Delta t \,f(u_n) + O(\Delta t)^2.
\end{equation}

Questa formula è chiamata metodo di Eulero. Per le equazioni di moto fugoide, il metodo di Eulero fornisce il seguente algoritmo:

\begin{align}
z_{n+1} & = z_n + \Delta t \, b_n \\
b_{n+1} & = b_n + \Delta t \left(g - \frac{g}{z_t} \, z_n \right).
\end{align}

La costante di integrazione scelta è il valore della derivata al tempo t = 0 (condizione iniziale) : 

\begin{equation}
u(t=0)=u_0
\end{equation}

    \newpage
    \subsubsection*{Fugoide Con Attrito}
    L’equazione per il moto fugoide con frizione è un sistema di equazioni differenziali ordinarie del primo ordine: 

\begin{align}
m \frac{dv}{dt} & = - W \sin\theta - D \\
m v \, \frac{d\theta}{dt} & = - W \cos\theta + L
\end{align}

Per visualizzare le traiettorie di volo previste da questo modello, che dipendono sia dalla velocità di avanzamento v sia dall’angolo della traiettoria theta. La posizione dell’aliante su un piano verticale sarà determinata dalle coordinate (x,y): 

\begin{align}
x'(t) & = v \cos(\theta) \\
y'(t) & = v \sin(\theta).
\end{align}


L’intero sistema di equazioni discretizzate con il metodo di Eulero è:

\begin{align}
v^{n+1} & = v^n + \Delta t \left(- g\, \sin\theta^n - \frac{C_D}{C_L} \frac{g}{v_t^2} (v^n)^2 \right) \\
\theta^{n+1} & = \theta^n + \Delta t \left(- \frac{g}{v^n}\,\cos\theta^n + \frac{g}{v_t^2}\, v^n \right) \\
x^{n+1} & = x^n + \Delta t \, v^n \cos\theta^n \\
y^{n+1} & = y^n + \Delta t \, v^n \sin\theta^n.
\end{align}

Scritto in forma vettoriale risulta: 

$$u'(t) = f(u)$$

\begin{align}
u & = \begin{pmatrix} v \\ \theta \\ x \\ y \end{pmatrix} & f(u) & = \begin{pmatrix} - g\, \sin\theta - \frac{C_D}{C_L} \frac{g}{v_t^2} v^2 \\ - \frac{g}{v}\,\cos\theta + \frac{g}{v_t^2}\, v \\ v\cos\theta \\ v\sin\theta \end{pmatrix}.
\end{align}

Le condizioni iniziali sono rappresentate dalle costanti di integrazione definite dal valore della derivata al tempo t = 0 :

$$
v(0) = v_0 \quad \text{and} \quad \theta(0) = \theta_0\\
x(0) = x_0 \quad \text{and} \quad y(0) = y_0
$$


    \newpage
    \subsubsection*{Convezione}
    L’equazione di convezione lineare unidimensionale è un’equazione differenziale alle derivate parziali: 

\begin{equation}
\frac{\partial u}{\partial t} + c \frac{\partial u}{\partial x} = 0
\end{equation}

Per la soluzione numerica di u(x,t) si sono utilizzati i pedici per denotare la posizione spaziale, come u$_i$, e gli apici per denotare l’istante temporale, come un : 

$$
\begin{matrix}
& &\bullet & & \bullet & &  \bullet \\
& &u^{n+1}_{i-1} & & u^{n+1}_i & & u^{n+1}_{i+1} \\
& &\bullet & & \bullet & &  \bullet \\
& &u^n_{i-1} & & u^n_i & & u^n_{i+1} \\
& &\bullet & & \bullet & &  \bullet \\
& &u^{n-1}_{i-1} & & u^{n-1}_i & & u^{n-1}_{i+1} \\
\end{matrix}
$$

L’equazione per fornire la soluzione numerica del problema è data da:

\begin{equation}
u_i^{n+1} = u_i^n - c \frac{\Delta t}{\Delta x}(u_i^n-u_{i-1}^n)
\end{equation}

Le condizioni iniziali per una funzione d’onda quadra sono definite così:

\begin{equation}
u(x,0)=\begin{cases}2 & \text{where } 0.5\leq x \leq 1,\\
1 & \text{everywhere else in } (0, 2)
\end{cases}
\end{equation}

dove il dominio della soluzione numerica è definito in \textit{x} $\in$(0,2).
Poniamo inoltre delle condizioni al contorno su \textit{x}: sia u = 1 quando \textit{x} = 0
    \newpage
    \subsubsection*{Burgers}
    L’equazione di Burgers unidimensionale è un’equazione differenziale alle derivate parziali: 

\begin{equation}
\frac{\partial u}{\partial t} + u \frac{\partial u}{\partial x} = \nu \frac{\partial ^2u}{\partial x^2}
\end{equation}

L’equazione per fornire la soluzione numerica del problema è data da: 

\begin{equation}
u_i^{n+1} = u_i^n - u_i^n \frac{\Delta t}{\Delta x} (u_i^n - u_{i-1}^n) + \nu \frac{\Delta t}{\Delta x^2}(u_{i+1}^n - 2u_i^n + u_{i-1}^n)
\end{equation}

dove u: 

\begin{equation}
- \frac{2\nu\left(-\frac{(-8t + 2x) e^{-\frac{(-4t + x)^2}{4\nu(t + 1)}}}{4\nu(t + 1)} - \frac{(-8t + 2x - 4\pi) e^{-\frac{(-4t + x - 2\pi)^2}{4\nu(t + 1)}}}{4\nu(t + 1)} \right)}{e^{-\frac{(-4t + x - 2\pi)^2}{4\nu(t + 1)}} + e^{-\frac{(-4t + x)^2}{4\nu(t + 1)}}} + 4
\end{equation}


e le condizioni iniziali sono definite con u(x, 0). Le condizioni al contorno sono: 

\begin{equation}
u(0) = u(2\pi)
\end{equation}

\section{Progettazione dell'Algoritmo}
\subsection{Scelte di Progetto}
    \subsubsection{Haskell}
    \subsubsection{Prolog}

\subsection{Passi dell'Algoritmo}

\section{Implementazione dell'algoritmo}

\subsection{Haskell}

\subsection{Prolog}

\section{Testing}

\subsection{Haskell}

\subsection{Prolog}

\end{document}
