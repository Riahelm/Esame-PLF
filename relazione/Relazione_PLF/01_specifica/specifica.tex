\section{Specifica del Problema}

Scrivere un programma Haskell e un programma Prolog che, per ogni simulazione numerica da effettuare, acquisiscono da tastiera un insieme finito di parametri numerici e poi stampano su schermo il risultato del calcolo numerico. Per  simulazione numerica intendiamo la simulazione di un processo fisico mediante calcolatore, ove con simulazione di un processo fisico si fa riferimento alla rappresentazione, eventualmente approssimata, di tale processo mediante la risoluzione di equazioni e modelli matematici al calcolatore \footnote{\url{https://www.treccani.it/enciclopedia/simulazioni-di-processi-fisici-mediante-calcolatore_(Enciclopedia-della-Scienza-e-della-Tecnica)/}}. Le simulazioni numeriche coinvolte sono quattro: integrazione di equazioni differenziali di moto fugoide senza attrito, di moto fugoide con attrito, di convezione lineare a una dimensione ed di Burgers a una dimensione. 
