\section{Analisi del Problema} 

\subsection{Dati di Ingresso del Problema}

I dati in ingresso al problema sono stati suddivisi in base all'equazione da integrare numericamente, ne segue quindi la loro descrizione.

\subsubsection*{Moto Fugoide Senza Attrito}
Il dato in ingresso è un numero reale maggiore di zero.
Questo rappresenta il passo temporale per l'integrazione dell'equazione del moto fugoide senza attrito.
\subsubsection*{Moto Fugoide Con Attrito}
Il dato in ingresso è un numero reale maggiore di zero. Questo rappresenta il passo temporale per l'integrazione dell'equazione del moto fugoide con attrito.
\subsubsection*{Moto di Convezione}
    I dati in ingresso per l'integrazione dell'equazione di convezione sono:
    \begin{enumerate}
        \item un numero naturale, rappresenta il numero di punti della funzione d'onda quadra;
        \item un numero reale maggiore di zero, rappresenta la lunghezza del passo temporale.
    \end{enumerate}
\subsubsection*{Moto di Convezione e Diffusione (Burgers)}
Il dato in ingresso per l'integrazione dell'equazione di Burgers (la quale viene utilizzata per risolvere il problema del moto di convezione e diffusione dell'onda) è un numero naturale. Questo rappresenta il numero di punti della funzione d'onda a dente di sega.

\subsection{Dati di Uscita del Problema}

\subsubsection*{Moto Fugoide Senza Attrito}
Il dato in uscita dell'integrazione dell'equazione del moto fugoide senza attrito è una sequenza di numeri reali che rappresentano la funzione di traiettoria dell'areomobile.
\subsubsection*{Moto Fugoide Con Attrito}
Il dato in uscita dell'integrazione dell'equazione del moto fugoide con attrito è una sequenza di numeri reali che rappresentano la funzione di traiettoria dell'areomobile.
\subsubsection*{Moto di Convezione}
Il dato in uscita dell'integrazione dell'equazione di convezione lineare a una dimensione è una sequenza di numeri reali che rappresentano i valori finali della funzione d'onda quadra.
\subsubsection*{Moto di Convezione e Diffusione (Burgers)}
Il dato in uscita all'integrazione dell'equazione di Burgers a una dimensione è una sequenza di numeri reali che rappresentano i valori finali della la funzione a dente di sega.

\subsection{Relazioni Intercorrenti tra i Dati del Problema}\label{analisi}

\subsubsection*{Moto Fugoide Senza Attrito}
L’equazione per il moto di fugoide senza attrito è un’equazione differenziale ordinaria del secondo ordine:

\begin{equation}
z(t)'' = g - \frac{g \,z(t)}{z_t} = g \left(1 - \frac{z(t)}{z_t}\right).
\end{equation}

\noindent
Possiamo trasformare questa equazione del secondo ordine in un sistema di equazioni del primo ordine:

\begin{equation}
z'(t) = b(t)
\end{equation}

\begin{equation}
b'(t) = g\left(1-\frac{z(t)}{z_t}\right).
\end{equation}

\noindent
Un altro modo di considerare un sistema di due equazioni ordinarie del primo ordine è scrivere il sistema differenziale come un’unica equazione vettoriale:

\begin{equation}
\vec{u}  = \begin{pmatrix} z \\ b \end{pmatrix}
\end{equation}

\begin{equation}
\vec{u}'(t)  = \begin{pmatrix} b\\ g-g\frac{z(t)}{z_t} \end{pmatrix}.
\end{equation}

\noindent
La soluzione approssimativa al tempo $t_n$ è $u_n$ e la soluzione numerica dell’equazione differenziale consiste nel calcolare una sequenza di soluzioni con la seguente equazione:

\begin{equation}
u_{n+1} = u_n + \Delta t \,f(u_n) + O(\Delta t)^2.
\end{equation}

\noindent
Questa formula è chiamata metodo di Eulero. Per le equazioni di moto fugoide, il metodo di Eulero fornisce il seguente algoritmo:

\begin{align}
z_{n+1} & = z_n + \Delta t \, b_n \\
b_{n+1} & = b_n + \Delta t \left(g - \frac{g}{z_t} \, z_n \right)
\end{align}

\noindent
dove:
\begin{itemize}
\item $\Delta t$ è la lunghezza del passo temporale;
\item $g$ è la forza gravitazionale terrestre;
\item $z_n$ è l'altitudine del velivolo al passo $n$;
\item $z_t$ è l'altitudine centrale del velivolo;
\item $b_n$ è la velocità del velivolo al passo $n$.
\end{itemize}

\noindent
Il numero di passi di simulazione $n$ viene calcolato $n = \frac{T}{\Delta t}$, dove $T$ è il
tempo totale di simulazione.

\noindent 
La condizione iniziale è il valore della derivata al tempo $t=0$ ed è l'altitudine iniziale del velivolo $z_0$.

\subsubsection*{Moto Fugoide Con Attrito}
L’equazione per il moto fugoide con attrito è un sistema di equazioni differenziali ordinarie del primo ordine: 

\begin{align}
m \frac{dv}{dt} & = - W \sin\theta - D \\
m v \, \frac{d\theta}{dt} & = - W \cos\theta + L.
\end{align}

\noindent
Per visualizzare le traiettorie di volo previste da questo modello, che dipendono sia dalla velocità di avanzamento $v$ sia dall’angolo della traiettoria $\theta$, occorre calcolare la posizione dell'aliante ad ogni istante di tempo $t$. La posizione dell’aliante su un piano verticale sarà determinata dalle coordinate $(x,y)$: 

\begin{align}
x'(t) & = v \cos(\theta) \\
y'(t) & = v \sin(\theta).
\end{align}

\noindent
L’intero sistema di equazioni discretizzate con il metodo di Eulero è:

\begin{align}
v^{n+1} & = v^n + \Delta t \left(- g\, \sin\theta^n - \frac{C_D}{C_L} \frac{g}{v_t^2} (v^n)^2 \right) \\
\theta^{n+1} & = \theta^n + \Delta t \left(- \frac{g}{v^n}\,\cos\theta^n + \frac{g}{v_t^2}\, v^n \right) \\
x^{n+1} & = x^n + \Delta t \, v^n \cos\theta^n \\
y^{n+1} & = y^n + \Delta t \, v^n \sin\theta^n.
\end{align}

\noindent
Scritto in forma vettoriale risulta: 

$$u'(t) = f(u)$$

\begin{align}
u & = \begin{pmatrix} v \\ \theta \\ x \\ y \end{pmatrix} & f(u) & = \begin{pmatrix} - g\, \sin\theta - \frac{C_D}{C_L} \frac{g}{v_t^2} v^2 \\ - \frac{g}{v}\,\cos\theta + \frac{g}{v_t^2}\, v \\ v\cos\theta \\ v\sin\theta \end{pmatrix}
\end{align}

\noindent
dove:
\begin{itemize}
\item $\Delta t$ è la lunghezza del passo temporale;
\item $g$ è la forza gravitazionale terrestre;
\item $x$ è lo spostamento orizzontale del velivolo;
\item $y$ è l'altitudine del velivolo;
\item $v$ è la velocità del velivolo;
\item $v_t$ è la velocità di trim;
\item $\theta$ è l'angolo d'inclinazione del velivolo;
\item $C_D$ è il coefficiente di resistenza dell'aria;
\item $C_L$ è il coefficiente di portanza.
\end{itemize}

\noindent
Il numero di passi di simulazione $n$ viene calcolato $n = \frac{T}{\Delta t}$, dove $T$ è il
tempo totale di simulazione.

\noindent
Le condizioni iniziali sono rappresentate dalle costanti di integrazione definite dal valore della derivata al tempo $t = 0$ :

$$
v(0) = v_0 \quad \text{and} \quad \theta(0) = \theta_0
$$
$$
x(0) = x_0 \quad \text{and} \quad y(0) = y_0.
$$

\subsubsection*{Moto di Convezione}
L’equazione di convezione lineare unidimensionale è un’equazione differenziale alle derivate parziali: 

\begin{equation}
\frac{\partial u}{\partial t} + c \frac{\partial u}{\partial x} = 0.
\end{equation}

\noindent
Per la soluzione numerica di $u(x,t)$ si sono utilizzati i pedici per denotare la posizione spaziale, come in $u_i$, e gli apici per denotare l’istante temporale, come in $u^n$ : 

$$
\begin{matrix}
& &\bullet & & \bullet & &  \bullet \\
& &u^{n+1}_{i-1} & & u^{n+1}_i & & u^{n+1}_{i+1} \\
& &\bullet & & \bullet & &  \bullet \\
& &u^n_{i-1} & & u^n_i & & u^n_{i+1} \\
& &\bullet & & \bullet & &  \bullet \\
& &u^{n-1}_{i-1} & & u^{n-1}_i & & u^{n-1}_{i+1}. \\
\end{matrix}
$$

\noindent
L’equazione per fornire la soluzione numerica del problema è data da:

\begin{equation}
u_i^{n+1} = u_i^n - c \frac{\Delta t}{\Delta x}(u_i^n-u_{i-1}^n)
\end{equation}

\noindent
dove:
\begin{itemize}
\item $\Delta t$ è la lunghezza del passo temporale;
\item $\Delta x$ è la lunghezza del passo spaziale;
\item $c$ è la velocità dell'onda.
\end{itemize}


\noindent
Le condizioni iniziali per una funzione d’onda quadra sono definite così:

\begin{equation}
u(x,0)=\begin{cases}2 & \text{dove } 0.5\leq x \leq 1,\\
1 & \text{altrimenti} 
\end{cases}
\end{equation}

\noindent
dove il dominio della soluzione numerica è definito in $x\in(0,2)$. In questo modo la lunghezza del passo spaziale viene calcolata come $\Delta x = \frac{|d|}{n - 1}$, dove $|d|$ è la distanza tra l'estremo inferiore e l'estremo superiore del dominio della soluzione numerica, $n$ il numero di punti della funzione d'onda discretizzata. Il numero totale di passi temporali $i$ è invece costante.
\noindent
Poniamo inoltre delle condizioni al contorno su $x$ in modo tale da ottenere il primo punto della funzione invariato per tutto il calcolo:

\begin{equation}
u(0,t) = c 
\end{equation}

\noindent
dove $c$ è una costante e il suo valore è pari al valore del primo punto della funzione.

\subsubsection*{Moto di Convezione e Diffusione (Burgers)}
L’equazione di Burgers unidimensionale è un’equazione differenziale alle derivate parziali che fonde in sè un'equazione di convezione e una di diffusione: 

\begin{equation}
\frac{\partial u}{\partial t} + u \frac{\partial u}{\partial x} = \nu \frac{\partial ^2u}{\partial x^2}.
\end{equation}

\noindent
L’equazione per fornire la soluzione numerica del problema è data da: 

\begin{equation}
u_i^{n+1} = u_i^n - u_i^n \frac{\Delta t}{\Delta x} (u_i^n - u_{i-1}^n) + \nu \frac{\Delta t}{\Delta x^2}(u_{i+1}^n - 2u_i^n + u_{i-1}^n)
\end{equation}

\noindent
dove: 
\begin{itemize}
\item $\nu$ è il coefficiente di diffusione dell'onda;
\item $u$:
\begin{equation} 
- \frac{2\nu\left(-\frac{(-8t + 2x) e^{-\frac{(-4t + x)^2}{4\nu(t + 1)}}}{4\nu(t + 1)} - \frac{(-8t + 2x - 4\pi) e^{-\frac{(-4t + x - 2\pi)^2}{4\nu(t + 1)}}}{4\nu(t + 1)} \right)}{e^{-\frac{(-4t + x - 2\pi)^2}{4\nu(t + 1)}} + e^{-\frac{(-4t + x)^2}{4\nu(t + 1)}}} + 4 
\end{equation}
\end{itemize}
.


\noindent
Il passo spaziale $\Delta x$ viene calcolato come precedentemente mostrato per l'equazione di convezione. Il passo temporale $\Delta t$ viene invece calcolato come $\Delta t = \frac{\sigma\Delta x^2}{n - 1}$, dove $\sigma$ è la
costante di Courant-Friedrichs-Lewy (CFL) e $n$ il numero di punti della funzione d'onda discretizzata. Il numero di passi temporali totali è $\frac{T}{\Delta t}$, dove $T$ è il tempo totale di simulazione.

\noindent
Le condizioni iniziali sono definite:

\begin{equation}
u(x, 0).
\end{equation}

\noindent
Le condizioni al contorno sono: 

\begin{equation}
u(0) = u(2\pi).
\end{equation}


